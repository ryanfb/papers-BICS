%
%  untitled
%
%  Created by Ryan Baumann on 2010-10-29.
%  Copyright (c) 2010 __MyCompanyName__. All rights reserved.
%
\documentclass[]{article}

% Use utf-8 encoding for foreign characters
\usepackage[utf8]{inputenc}

% Setup for fullpage use
\usepackage{fullpage}

% Uncomment some of the following if you use the features
%
% Running Headers and footers
%\usepackage{fancyhdr}

% Multipart figures
%\usepackage{subfigure}

% More symbols
%\usepackage{amsmath}
%\usepackage{amssymb}
%\usepackage{latexsym}

% Surround parts of graphics with box
\usepackage{boxedminipage}

% Package for including code in the document
\usepackage{listings}

% If you want to generate a toc for each chapter (use with book)
\usepackage{minitoc}

% This is now the recommended way for checking for PDFLaTeX:
\usepackage{ifpdf}

\usepackage{hyperref}
\hypersetup{
colorlinks = false,
pdfborder = 0 0 0,
pdftitle = {The Son of Suda On-Line},
plainpages = false,
}

\usepackage{natbib}
\bibpunct{(}{)}{;}{a}{}{, }
% \usepackage{inlinebib}

%\newif\ifpdf
%\ifx\pdfoutput\undefined
%\pdffalse % we are not running PDFLaTeX
%\else
%\pdfoutput=1 % we are running PDFLaTeX
%\pdftrue
%\fi

\ifpdf
\usepackage[pdftex]{graphicx}
\else
\usepackage{graphicx}
\fi

\usepackage{fontspec,xunicode}
\setmainfont{Adobe Garamond Pro}
\setmonofont{Gill Sans}

\title{The Son of Suda On-Line}
\author{Ryan Baumann}

\date{}

\begin{document}

\ifpdf
\DeclareGraphicsExtensions{.pdf, .jpg, .tif}
\else
\DeclareGraphicsExtensions{.eps, .jpg}
\fi

\maketitle

\section*{Introduction}

Integrating Digital Papyrology (IDP) is a joint project among a number of institutions aimed at improving and combining the relationships between three digital papyrological resources: the Duke Databank of Documentary Papyri (DDbDP), the Heidelberger Gesamtverzeichnis der griechischen Papyrusurkunden \"{A}gyptens (HGV), and the Advanced Papyrological Information System (APIS). Started in 1983, the DDbDP collects a number of transcriptions of ancient documentary papyri from print editions. HGV and APIS, meanwhile, collect metadata (findspots, dates, etc.) and images of much of the same material. The benefits of unifying these data sources should be obvious, and in 2007 the Scholarly Communications Division of the Andrew W. Mellon Foundation funded a project, “Integrating Digital Papyrology,” to do just that. Over the years the DDbDP has undergone a number of transitions, and this project also supported its transition from idiosyncratic SGML to standards-based EpiDoc XML \citep{background, sosin}.

\begin{quote}
  \begin{enumerate}
    \item Improve operability of the PN search interface on the merged and mapped data from the DDBDP, HGV, and APIS
    \item Facilitate third-party use of the data and tools
    \item Create a version controlled, transparent and fully audited, multi-author, web-based, real-time, tagless, editing environment, which — in tandem with a new editorial infrastructure — will allow the entire community of papyrologists to take control of the process of populating these communal assets with data
  \end{enumerate}
\end{quote}\citet{sosin}

Test\footnote{\url{http://idp.atlantides.org/trac/idp/wiki/BackgroundAndFunding}} \cite{babeu}

\bibliographystyle{plainnat}
\bibliography{bics-sosol}
\end{document}
